\section{Описание}
Требуется написать реализацию алгоритма сортировки подсчётом.
Для последовательности ключей основная идея сортировки подсчётом состоит в том, чтобы подсчитать, сколько раз встретится каждый ключ, и затем вывести отсортированную последовательность. 

Для сложных структур типа \enquote{ключ-значение} алгоритм немного сложнее.

Для них, как сказано в \cite{cs_ifmo}, идея алгоритма состоит в предварительном подсчете количества элементов с различными ключами в исходном массиве и разделении результирующего массива на части соответствующей длины (будем называть их блоками). Затем при повторном проходе исходного массива каждый его элемент копируется в специально отведенный его ключу блок, в первую свободную ячейку. Это осуществляется с помощью массива индексов $P$, в котором хранятся индексы начала блоков для различных ключей. $P[key]$ — индекс в результирующем массиве, соответствующий первому элементу блока для ключа $key$. 

\pagebreak

\section{Исходный код}
На каждой непустой строке входного файла располагается пара \enquote{ключ-значение}, поэтому создадим новую 
структуру $Token$, в которой будем хранить ключ и значение.

Реализуем функцию $stringAssignment(source, destination)$, которая будет присваивать строке $destination$ содержимое строки $source$. 

Создаём массив $A$ и в цикле заполняем его. В процессе заполнения массива также пересчитываем количество структур, ключ которых равен $i$ и записываем это число в $keys[i]$.

Превращаем массив $keys$ в массив, хранящий в $keys[i]$ сумму элементов от $0$ до $i - 1$ старого массива $keys$, а затем \enquote{сдвигаем} его на элемент вправо, так, чтобы получился массив, содержащий в $keys[i]$ индекс начала блока ключа $i$ в массиве $B$.

Произведем саму сортировку. Еще раз пройдем по исходному массиву A и для всех $i\in[0, n - 1]$ будем помещать структуру $A[i]$ в массив B на место $keys[A[i].key]$, а затем увеличивать $keys[A[i].key]$ на $1$. Здесь $A[i].key$ — это ключ структуры, находящейся в массиве $A$ на $i$-том месте. 

После завершения алгоритма в $B$ будет содержаться отсортированная исходная последовательность.

\begin{lstlisting}[language=C++]
#include <iostream>

struct Token {
	int key;
	char data[2048];
};

int stringAssignment(char* source, char* destination) {
	int i = 0;
	while (source[i] != '\0') {
		destination[i] = source[i];
		++i;
	}
	destination[i] = source[i];
	return 0;
}

int main() {
	int keys[1000000];
	for (int i = 0; i < 1000000; ++i) {
		keys[i] = 0;
	}
	int num_count = 0;
	int int_buff = 0;
	char char_buff[2048];
	Token* array_a = nullptr;

	while (scanf("%d\t%s", &int_buff, char_buff) > 0) {
		++keys[int_buff];
		++num_count;
		array_a = (Token*)realloc(array_a, sizeof(Token) * num_count);
		array_a[num_count - 1].key = int_buff;
		stringAssignment(char_buff, array_a[num_count - 1].data);
	}

	Token* array_b = (Token*)malloc(sizeof(Token) * num_count);
	for (int i = 0; i < num_count; ++i) {
		array_b[i].key = -1;
		for (int j = 0; j < 2048; ++j) {
			array_b[i].data[j] = '\0';
		}
	}

	for (int i = 1; i < 1000000; ++i) {
		keys[i] = keys[i] + keys[i - 1];
	}

	for (int i = 999999; i > 0; --i) {
		keys[i] = keys[i - 1];
	}

	keys[0] = 0;

	for (int i = 0; i < num_count; ++i) {
		array_b[keys[array_a[i].key]].key = array_a[i].key;
		stringAssignment(array_a[i].data, array_b[keys[array_a[i].key]].data);
		++keys[array_a[i].key];
	}

	for (int i = 0; i < num_count; ++i) {
		printf("%06d\t%s\n", array_b[i].key, array_b[i].data);
	}

	free(array_a);
	free(array_b);

	return 0;
}
	
\end{lstlisting}

\pagebreak

\section{Консоль}
\begin{alltt}
[alext@alext-pc solution]$ make
g++ -std=c++14 -pedantic -Wall -Wextra-Wno-unused-variable lab1.cpp -o solution
[alext@alext-pc solution]$ cat test1 
472891 asakdhfl
130391 bfsadfkjlsdf
891767 csdafKHdf
130000 dkhjs32
[alext@alext-pc solution]$ ./solution <\ test1 
130000	dkhjs32
130391	bfsadfkjlsdf
472891	asakdhfl
891767	csdafKHdf
\end{alltt}
\pagebreak