\section{Тест производительности}

Тест производительности представляет собой сравнение реализованного мной дерева с {\itshape std::map}, в котором ключами будут являться {\itshape std::string}, а данные -- {\itshape unsigned long long}.\\
Для вставки в {\itshape std::map} используется метод {\itshape insert()}, для удаления -- {\itshape erase()}. Для поиска будем использовать метод {\itshape find()}. 
В {\itshape std::map} не реализованы запись и чтение из файла, поэтому тестироваться будут только операции вставки, удаления и поиска.\\
Тесты состоят из $10^3$, $10^4$ и $10^5$ строк. Все ключи длиной 8 символов.

\begin{alltt}
[alext@alext-pc solution]$ g++ bc.cpp 
[alext@alext-pc solution]$ ./a.out < test1k.txt 
B-tree: 2.519 ms
std::map: 1.135 ms
[alext@alext-pc solution]$ ./a.out < test10k.txt 
B-tree: 31.711 ms
std::map: 11.747 ms
[alext@alext-pc solution]$ ./a.out < test100k.txt 
B-tree: 466.771 ms
std::map: 155.924 ms
\end{alltt}

{\itshape std::map} оказался примерно в 3 раза быстрее моего алгоритма. Но, тем не менее, мне удалось реализовать структуру В-дерева так, чтобы сложность была на том уровне, на котором она должна быть, и как, мы видим, время работы моего алгоритма всегда в 3 раза больше, чем у {\itshape std::map}, а значит их сложности совпадают.

\pagebreak

