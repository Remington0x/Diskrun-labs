\section{Тест производительности}

Тест производительности представляет собой сравнение реализованного мной дерева с {\itshape std::map}, в котором ключами будут являться {\itshape std::string}, а данные -- {\itshape unsigned long long}.\\
Для вставки в {\itshape std::map} используется метод {\itshape insert()}, для удаления -- {\itshape erase()}. Для поиска будем использовать метод {\itshape find()}. 
В {\itshape std::map} не реализованы запись и чтение из файла, поэтому тестироваться будут только операции вставки, удаления и поиска.\\
Тесты состоят из $10^3$, $10^4$ и $10^5$ строк. Все ключи длиной 8 символов.

\begin{alltt}
    [alext@alext-pc solution]$ ./benchmark < test1k.txt
    B-tree: 39.656 ms
    std::map: 1.067 ms
    [alext@alext-pc solution]$ ./benchmark < test10k.txt
    B-tree: 2547.392 ms
    std::map: 14.117 ms
    [alext@alext-pc solution]$ ./benchmark < test100k.txt
    B-tree: 446774.381 ms
    std::map: 368.464 ms
\end{alltt}

{\itshape std::map} оказался гораздо быстрее моего алгоритма. Также очевидно, что сложность моего алгоритма больше чем у {\itshape std::map}. Я не знаю, почему так происходит. В процессе написания лабораторной я тщательно сверялся с описанием максимально эффективных реализаций операций, но это не помогло качественно реализовать структуру B-дерева.

\pagebreak

