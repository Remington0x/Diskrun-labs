\section{Описание}
Требуется осуществить реализацию B-дерева, реализовать функции добавления в него ключей, поиска по дереву, удаления ключей. Также необходимо реализовать запись структуры в файл и чтение из него.\\
Согласно \cite{b-tree-ifmo}, B-дерево является идеально сбалансированным, то есть глубина всех его листьев одинакова. B-дерево имеет следующие свойства ($t$ — параметр дерева, называемый минимальной степенью B-дерева, не меньший $2$.):
\begin{itemize}
	\item Каждый узел, кроме корня, содержит не менее $t-1$ ключей, и каждый внутренний узел имеет по меньшей мере $t$ дочерних узлов. Если дерево не является пустым, корень должен содержать как минимум один ключ. 
	\item Каждый узел, кроме корня, содержит не более $2t-1$ ключей и не более чем $2t$ сыновей во внутренних узлах.
	\item Корень содержит от 1 до $2t-1$ ключей, если дерево не пусто и от $2$ до $2t$ детей при высоте большей 0.
	\item Каждый узел дерева, кроме листьев, содержащий ключи $k1,...,kn$, имеет $n+1$ сына. $i$-й сын содержит ключи из отрезка $[k_{i-1};k_{i}]$, $k_0=-\infty$, $k_n+1=\infty$.
	\item Ключи в каждом узле упорядочены по неубыванию.
	\item Все листья находятся на одном уровне.
\end{itemize}

\pagebreak

\section{Исходный код}
Поскольку в качестве ключа будет использоваться строка символов, нам пригодится реализация string. Создаём структуру {\ttfamily TString}, реализуем для неё некоторые операторы, в частности операторы {\itshape сравнения}.\\
Изначально ноды дерева были реализованы в виде массивов, но это привело к ошибкам и утечкам памяти, в связи с чем было принято решение реализовать списки ключей и данных в нодах при помощи структуры {\ttfamily vector}. Реализую методы, схожие с методами {\ttfamily std::vector: at(), erase(), insert()} и т.\ д.\\
В файле основной программы создадим класс {\ttfamily TBTree}, реализуем в нём структуру ноды {\ttfamily TNode}, необходимые поля, а также различные методы, которые будут перечислены в таблице далее.\\
{\ttfamily main()} по большей части состоит из т.\ н. игрового цикла, в котором до конца ввода обрабатываются команды пользователя.

\begin{lstlisting}[language=C++]
template <class T>
class TVectorNode {
public:
	T data;
	std::shared_ptr< TVectorNode<T> > next;
	TVectorNode() noexcept : data(), next(nullptr) {};
	explicit TVectorNode(const T & elem) noexcept : data(elem), next(nullptr) {};
};

template <class T>
class TVector {
private:
	int size;
	std::shared_ptr< TVectorNode<T> > leftNode;
public:
	void insert(T data, int pos);
	T at(int pos);
	void erase(int pos);
	void erase(int pos1, int pos2);
	int getSize();

	TVector() noexcept : leftNode(nullptr) {};
};
\end{lstlisting}

\begin{longtable}{|p{7.5cm}|p{7.5cm}|}
\hline
\rowcolor{lightgray}
\multicolumn{2}{|c|} {tvector.hpp}\\
\hline
{\ttfamily void insert(T data, int pos)}&Вставляет {\ttfamily data} на позицию {\ttfamily pos}.\\
\hline
{\ttfamily T at(int pos)}&Возвращает элемент, находящийся на позиции {\ttfamily pos}.\\
\hline
{\ttfamily void erase(int pos)}&Удаляет элемент с позиции {\ttfamily pos}.\\
\hline
{\ttfamily void erase(int pos1, int pos2)}&Удаляет элементы с позиции {\ttfamily pos1} до {\ttfamily pos2}.\\
\hline
{\ttfamily int getSize()}&Возвращает размер вектора.\\
\hline
\end{longtable}

\begin{lstlisting}[language=C++]
	struct TString {
    char string[MAX_LENGTH];
    TString() {
        string[0] = '\0';
    }
    TString(const TString& s) {
        for (int i = 0; i < 256; ++i) {
            string[i] = s.string[i];
        }
    }
    ~TString() {}
    void nullStr() {
        for (int i = 0; i < MAX_LENGTH; ++i) {
            string[i] = '\0';
        }
    }
    TString& operator=(const TString& b);
};
\end{lstlisting}

\begin{longtable}{|p{7.5cm}|p{7.5cm}|}
\hline
\rowcolor{lightgray}
\multicolumn{2}{|c|} {tstring.hpp}\\
\hline
{\ttfamily TString()}&Конструктор по умолчанию, кладёт в нулевой элемент строки терминальный символ.\\
\hline
{\ttfamily TString(const TString\& s)}&Конструктор, инициализирующий новую строку уже существующей.\\
\hline
{\ttfamily void nullStr()}&Метод, заполняющий всю строку терминальными символами.\\
\hline
{\ttfamily bool operator<(const TString\& a, const TString\& b)}&Перегрузка оператора \enquote{меньше}.\\
\hline
{\ttfamily bool operator>(const TString\& a, const TString\& b)}&Перегрузка оператора \enquote{больше}.\\
\hline
{\ttfamily bool operator==(const TString\& a, const TString\& b)}&Перегрузка оператора \enquote{равно}.\\
\hline
{\ttfamily std::ostream\& operator$<<$(std::ostream\& out, const TString\& a)}&Перегрузка оператора вывода.\\
\hline
\end{longtable}

В этом случае структуры или классы должны быть полностью приведены в листинге (без реализации методов).
\begin{lstlisting}[language=C++]
class TBTree {
private:
	struct TNode {
		bool isLeaf;
		int keyCount;
		TVector<TString> keys;
		TVector<unsigned long long> data;
		TVector<TNode*> children;
		TNode* parent;
	};
	TNode* root;
	bool isSuchWord;
	bool doesAlreadyExist;
	int t;
	int nodeCount;

	template <class T>
	bool deleteFromArray(T* array, int size, int pos);
	unsigned long long recSearch(TNode* node, TString& key);
	void nodeAddKey(TNode* node, TString& key, unsigned long long& data);
	void recCheckParentage(TNode* node);
	int recCheckCounter;
	void _rCS(TNode* node);
	void recCheckSize(TNode* node);
	TNode* nodeSplit(TNode* node, TNode* child);
	void recInsert(TNode* node, TString& key, unsigned long long& data);
	void recFreeTree(TNode* node);
	void recPrintTree(TNode* node, int tab);
	void recSave(TNode* node, FILE* out);
	bool loadNode(TNode* node, TNode* parent, FILE* in);
	bool deleteKey(TNode* node, const TString& key);
	TNode* lookForBros(TNode* node);
	int nodeFill(TNode* node);
	TNode* nodeMerge(TNode* nodeOrig, TNode* bratelnikOrig);
	TNode* findRight(TNode* node);
	TNode* findLeft(TNode* node);
	bool recDelete(TNode* node, const TString& key);

public:
	void init(int treeMeasure);
	TBTree();
	TBTree(int treeMeasure);
	~TBTree();
	void search(TString& key);
	bool insert(TString& key, unsigned long long& data);
	void printTree();
	void save(FILE* out);
	bool load(FILE* in);
	int getMeasure();
	bool deleteKeyFromTree(TString& key);
	void checkParentage();
};
\end{lstlisting}

\begin{longtable}{|p{7.5cm}|p{7.5cm}|}
\hline
\rowcolor{lightgray}
\multicolumn{2}{|c|} {main.cpp}\\
\hline
{\ttfamily unsigned long long recSearch(TNode* node, TString\& key)}&Рекуррентная функция, вспомогательная для публичного метода {\ttfamily search}. Возвращает значение, соответствующее ключу {\ttfamily key}, начиная поиск в ноде {\ttfamily node}.\\
\hline
{\ttfamily void nodeAddKey(TNode* node, TString\& key, unsigned long long\& data)}&Добавляет {\ttfamily key} и {\ttfamily data} в узел {\ttfamily node}.\\
\hline
{\ttfamily void recCheckParentage(TNode* node)}&Рекуррентная функция проверки правильности информации о родителе. Используетя в отладочных целях.\\
\hline
{\ttfamily void recCheckSize(TNode* node)}&Проверяет соответствие атрибута узла реальному размеру. Используется в отладочных целях.\\
\hline
{\ttfamily TNode* nodeSplit(TNode* node, TNode* child)}&Разбивает узел размера $2t-1$ на два узла размерами $t-1$.\\
\hline
{\ttfamily void recInsert(TNode* node, TString\& key, unsigned long long\& data)}&Рекуррентная функция, выполняет вставку значений {\ttfamily key} и {\ttfamily data} в дерево. Вспомогательная функция для метода {\ttfamily insert}\\
\hline
{\ttfamily void recFreeTree(TNode* node)}&Корректно освобождает занятую деревом память.\\
\hline
{\ttfamily void recPrintTree(TNode* node, int tab)}&Рекуррентная функция печати дерева на экран. Вспомогательная для {\ttfamily printTree}.\\
\hline
{\ttfamily void recSave(TNode* node, FILE* out)}&Рекуррентная функция сохранения узла {\ttfamily node} в файл {\ttfamily out}. Вспомогательная для {\ttfamily TBTree::save}\\
\hline
{\ttfamily bool loadNode(TNode* node, TNode* parent, FILE* in)}&Рекуррентная функция загрузки узла из файла, на вход принимаются указатель на область памяти, выделенную под узел, указатель на родительский узел, а также указатель на файл. Возвращает {\ttfamily true}, если считано успешно, {\ttfamily false} в обратном случае.\\
\hline
{\ttfamily bool deleteKey(TNode* node, const TString\& key)}&Удаляет из узла {\ttfamily node} значение {\ttfamily key} и соответствующее ему значение данных. Возвращает {\ttfamily true} в случае успеха, {\ttfamily false} в обратном случае.\\
\hline
{\ttfamily TNode* lookForBros(TNode* node)}&Ищет братьев, из которых можно взять значение для узла, который нужно дополнить. Возвращает указатель на подходящего брата, если он есть, или {\ttfamily nullptr} в обратном случае.\\
\hline
{\ttfamily int nodeFill(TNode* node)}&Дополняет узел, чтобы из него можно было удалить значение. Возвращает 0 в случае успеха, 1 в случае если целевая нода является корнем, 2 в случае если дополнить из братьев невозможно.\\
\hline
{\ttfamily TNode* nodeMerge(TNode* nodeOrig, TNode* bratelnikOrig)}&Выполняет слияние нод {\ttfamily nodeOrig} и {\ttfamily bratelnikOrig}. Возвращает указатель на ноду, которая получилась в результате.\\
\hline
{\ttfamily TNode* findRight(TNode* node)}&Рекуррентная функция для нахождения самого правого узла. Используется для нахождения правого узла левого поддерева.\\
\hline
{\ttfamily TNode* findLeft(TNode* node)}&Аналогичная функция, но для самого левого узла.\\
\hline
{\ttfamily bool recDelete(TNode* node, const TString\& key)}&Рекуррентная функция для удаления значения {\ttfamily key} начиная с ноды {\ttfamily node}.\\
\hline
{\ttfamily TBTree(int treeMeasure);}&Конструктор класса, создаёт дерево с параметром {\ttfamily treeMeasure}\\
\hline
{\ttfamily void search(TString\& key)}&Функция поиска ключа {\ttfamily key} в дереве.\\
\hline
{\ttfamily bool insert(TString\& key, unsigned long long\& data)}&Функция вставки ключа {\ttfamily key} и данных {\ttfamily data} в дерево. Возвращает {\ttfamily true}, если вставка успешна, {\ttfamily false} в случае неуспеха.\\
\hline
{\ttfamily void printTree()}&Выводит дерево на экран, может использоваться только на малом наборе данных в дереве, из-за возникающей путаницы при отображении.\\
\hline
{\ttfamily void save(FILE* out)}&Сохраняет дерево в файл {\ttfamily out}.\\
\hline
{\ttfamily bool load(FILE* in)}&Загружает дерево из файла {\ttfamily in}. Возвращает {\ttfamily true} в случае успеха, {\ttfamily false} в случае неуспеха.\\
\hline
{\ttfamily int getMeasure()}&Возвращает параметр дерева $t$.\\
\hline
{\ttfamily bool deleteKeyFromTree(TString\& key)}&Удаляет из дерева ключ {\ttfamily key}. Возвращает {\ttfamily true} в случае успеха, {\ttfamily false} в случае неуспеха.\\
\hline
{\ttfamily void checkParentage()}&Функция корректности информации о родителях, используется в отладочных целях.\\
\hline
\end{longtable}

\pagebreak

\section{Консоль}
\begin{alltt}
	[alext@alext-pc solution]$ make -B
	g++ -std=c++14 -pedantic -g -Wall -Wextra -Wno-unused-variable lab2-nocomms.cpp -o solution
	[alext@alext-pc solution]$ cat test.txt 
	+ Alpha 1239871
	! Save tree.b
	- Alpha
	Alpha
	+ Bravo 28293024
	+ Charlie 308094108
	Delta
	! Load tree.b
	+ Charlie 308094108
	+ Bravo 28293024
	Alpha
	Charlie
	Echo
	Bravo
	Foxtrot
	[alext@alext-pc solution]$ ./solution < test.txt 
	OK
	OK
	OK
	NoSuchWord
	OK
	OK
	NoSuchWord
	OK
	OK
	OK
	OK: 1239871
	OK: 308094108
	NoSuchWord
	OK: 28293024
	NoSuchWord
\end{alltt}
\pagebreak

