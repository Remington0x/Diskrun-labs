\section{Выводы}
В процессе выполнения этой лабораторной работы я тщательно разобрался со структурой В-дерева. Изначально в узлах списки ключей и данных хранились в массивах, но это приводило к колоссальным утечкам памяти, которые я не в силах был устранить. Пришлось реализовать структуру {\itshape TVector} и переписать код с нуля для того, чтобы гарантировано избежать ошибок в логике. От утечек памяти удалось избавиться полностью, но сама реализация получилась неудачной, как по сложности, так и по тому, что около 1\% узлов теряется при удалении из дерева.
\pagebreak
