\section{Valgrind}
Как сказано в \cite{man-valgrind}: \enquote{Valgrind -- гибкая программа для дебаггинга и профилирования исполняемых файлов Linux}.\\
Valgrind имеет много применений, но я использовал его для проверки корректной работы с памятью: её выделение, освобождение и обращение к ней.\\
Изначально программа выдавала множество ошибок по памяти, и не выдавала правильного решения. Функции тщательно отлаживались, а часто переписывались сначала, это привело к тому, что память перестала утекать, а программа начала выдавать правильные ответы.

\begin{alltt}
	[alext@alext-pc solution]$ valgrind --leak-check=full ./solution < test1k.txt > /dev/null
	==7598== Memcheck, a memory error detector
	==7598== Copyright (C) 2002-2017, and GNU GPL'd, by Julian Seward et al.
	==7598== Using Valgrind-3.17.0 and LibVEX; rerun with -h for copyright info
	==7598== Command: ./solution
	==7598== 
	==7598== 
	==7598== HEAP SUMMARY:
	==7598==     in use at exit: 0 bytes in 0 blocks
	==7598==   total heap usage: 3,497 allocs, 3,497 frees, 346,392 bytes allocated
	==7598== 
	==7598== All heap blocks were freed -- no leaks are possible
	==7598== 
	==7598== For lists of detected and suppressed errors, rerun with: -s
	==7598== ERROR SUMMARY: 0 errors from 0 contexts (suppressed: 0 from 0)
\end{alltt}

\pagebreak

\section{GPROF}
Согласно \cite{man-gprof}, gprof производит профиль исполнения программ на С, Pascal и Fortran77. Эта утилита используется для измерения времени работы отдельных функций программы и общего времени работы.\\
Профайлер показывает, сколько раз выполнялась конкретная функция и какое время от общего времени работы программы она заняла.\\
Тестирование проводилось на тесте из 1000 строк с запросами на добавление, поиск и удаление.
\begin{alltt}
	[alext@alext-pc solution]$ gprof ./solution -p ./gmon.out
Flat profile:

Each sample counts as 0.01 seconds.
  %   cumulative   self              self     total           
 time   seconds   seconds    calls  us/call  us/call  name    
 50.02      0.01     0.01   538093     0.02     0.02  std::_Sp_counted_base\\<(__gnu_cxx::_Lock_policy)2>::_M_add_ref_copy()
 50.02      0.02     0.01    78715     0.13     0.13  operator==\\(TString const&, TString const&)
  0.00      0.02     0.00   554500     0.00     0.00  std::_Sp_counted_base<\\(__gnu_cxx::_Lock_policy)2>::_M_release()
  0.00      0.02     0.00   366253     0.00     0.00  std::__shared_ptr<TVectorNode\\<TString>, (__gnu_cxx::_Lock_policy)2>::get() const
	[...]\\
	[Дальнейший лог было решено не приводить, поскольку он не несёт какой-либо особенно ценной информации и в то же время является слишком перегруженным для того, чтобы его было комфортно читать.]
\end{alltt}
Видно, что большую часть времени занимают операции работы с указателями и реализация оператора сравнения для строк.
\pagebreak

\section{Дневник отладки}
В общем на работу над лабораторной \textnumero 2 ушло более 3-х недель, дневник отладки я не вёл, так как отлаживал при помощи valgrind ещё до того, как узнал про эту лабораторную. \\
В процессе отладки были выявлены и устранены (возможно частично) следующие ошибки: 
\begin{itemize}
	\item Утечки памяти при malloc() и realloc(). Решена реализацией структуры {\ttfamily Vector} и использованием её вместо массива в узле дерева.
	\item Нарушение структуры \enquote{родитель-ребёнок}, когда в ребёнке оставался указатель не на его родителя, а на \enquote{дядю} (узел-брат для родителя), после разбиения вершины. Решена исправлением соответсвующих функций.
	\item Потеря части узлов при удалении лишь одного. Была {\itshape частично} решена корректировкой соответсвующих функций, но тем не менее, некоторые узлы по-прежнему теряются, хоть и в гораздо меньших масштабах, чем изначально.
\end{itemize}

\pagebreak


