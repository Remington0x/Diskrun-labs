\section{Тест производительности}

Тест производительности представляет из себя следующее: поиск образцов с помощью алгоритма Ахо-Корасик и с помощью наивного алгоритма. Образец достаточно мал (5 чисел) по сравнению с текстом (2500, 5000 и 10000 строк). Последний тест представляет собой длинный образец с джокерами, текст объёмом 1000 строк, вхождение образца на каджом числе.

\begin{alltt}
[alext@alext-pc tests]$ ./solution < 2,5k-normal.t 
Aho-Corasick algorithm: 35050us
Naive algorithm: 12067us
[alext@alext-pc tests]$ ./solution < 5k-normal.t 
Aho-Corasick algorithm: 70024us
Naive algorithm: 24251us
[alext@alext-pc tests]$ ./solution < 10k-normal.t 
Aho-Corasick algorithm: 139041us
Naive algorithm: 47547us
[alext@alext-pc tests]$ ./solution < 1k-same.t
Aho-Corasick algorithm: 219538us
Naive algorithm: 2337851us
\end{alltt}

Видно, что оба алгоритма соответствуют заявленной сложности, при размере образца много меньшем размера текста наивный алгоритм работает почти за линейное время, что и видно из тестов. За счёт простоты реализации и при таком размере образца он выигрывает у алгоритма Ахо-Корасик в абсолютном времени. Если сделать образец больше, а количество вхождений, пусть даже неполных, чаще, наивный алгоритм будет уже не так эффективен. Это видно в последнем тесте, где для наивного алгоритма представлен один из худших случаев работы. В нём алгоритму Ахо-Корасик удалось выиграть даже в абсолютном времени, за счёт того, что его сложность составляет $O(m + n + a)$, в отличие от наивного, чья сложность равна $O(m*(n - m))$, где m --- длина образца, n --- длина текста, a --- количество появлений подстрок шаблона.

\pagebreak

