\section{Выводы}
Применение алгоритмов поиска подстрок в строке чрезвычайно широко распространено. Они используются в любых приложениях, выполняющих обработку текстовых данных.

В лабораторной мне было предложено решить задачу поиска подстроки в строке при помощи модификации алгоритма Ахо-Корасик для образца с джокерами.

Алгоритм Ахо-Корасик представляется в среднем достаточно эффективным, но у него есть как свои достоинства, так и недостатки. 

К последним прежде всего относится его сложность к восприятию и усвоению принципа его работы. А также то, что он требует O(n) памяти, в отличие от наивного алгоритма, который занимает O(1) памяти.

К достоинствам алгоритма можно отнести стабильную невысокую вычислительную сложность.

Алгоритм Ахо-Корасик выгоден, когда есть список длинных, сложных и постоянных паттернов, и их необходимо искать во множестве разных текстов. В таком случае достаточно один раз построить бор и далее искать паттерны в тексте за один проход.

Когда же паттерны часто изменяются и/или много короче по сравнению с текстом, есть смысл использовать наивный алгоритм ввиду его простоты и приемлемой эффективности.
\pagebreak
